\begin{problem}[Engel Problem-Solving Strategies 1.10]
The vertices of an $n$-gon are labeled by real numbers $x_1, \dots, x_n$. Let $a, b, c, d$ be four successive labels. If $(a - d)(b - c) < 0$, then we may switch $b$ with $c$. Decide if this switching operation can be performed infinitely often.
\end{problem}

It cannot.

Consider the sum $$x_1x_2 + x_2x_3 + \dots + x_{n-1}x_n + x_nx_1.$$ Performing the switching operation only affects one part of this sum. Specifically, $$ab + bc + cd$$ becomes $$ac + cb + bd.$$ But we have
\begin{align*}
(a-d)(b-c) &< 0 \\
ab - ac - bd + cd &< 0 \\
ab + cd &< ac + bd,
\end{align*}
so our sum has increased. Since there are only finitely many orderings of labels, these increases cannot go on forever. Therefore, we must eventually eventually run out of legal switches.